% !TEX TS-program = xelatex
% !TEX encoding = UTF-8 Unicode
% !BIB TS-program = biber

\documentclass{beamer}
\usepackage[cm-default,no-math]{fontspec}
\usepackage{xunicode}
\usepackage{xltxtra}
\usepackage[utf8x]{inputenc}
\usepackage{listings}
\usepackage{hyperref} 
\usepackage{adjustbox}
\usepackage{tcolorbox}
\usepackage{pdfpages}
\usepackage{multicol}
\usepackage{cancel}
\usepackage[autostyle=true,german=quotes]{csquotes}
\usepackage{pifont}
\usepackage{color}
\usepackage{qrcode}

\usetheme{m} % Use metropolis theme


\definecolor{pblue}{rgb}{0.13,0.13,1}
\definecolor{pgreen}{rgb}{0,0.5,0}
\definecolor{pred}{rgb}{0.9,0,0}
\definecolor{pgrey}{rgb}{0.46,0.45,0.48}

\usepackage{listings}
\lstset{language=Java,
  frame=single,
  showspaces=false,
  showtabs=false,
  breaklines=true,
  numbers=left,
  showstringspaces=false,
  breakatwhitespace=true,
  commentstyle=\color{pgreen},
  keywordstyle=\color{pblue},
  tabsize=4,
  xleftmargin=8pt,
  stringstyle=\color{pred},
  basicstyle=\footnotesize\ttfamily,
  moredelim=[il][\textcolor{pgrey}]{$$},
  moredelim=[is][\textcolor{pgrey}]{\%\%}{\%\%}
}

\setbeamertemplate{itemize items}[square]
\setbeamercovered{transparent}

%% Commands

\newcommand{\code}[1]{\texttt{#1}}

\newcommand{\listing}[1]{
	\begin{itemize}
		\item[]\lstinputlisting[]{listings/#1}
	\end{itemize}
}

\graphicspath{ {./images/} }
\newcommand{\myfig}[2]{
	\begin{minipage}[c]{\textwidth}
		\begin{center}
			\includegraphics[keepaspectratio,width=#2\textwidth]{#1}
		\end{center}
		\vspace{3mm}
	\end{minipage}
}

\newcommand{\bb}[1]{\textbf{#1}}

\newcommand{\slideItems}[1]{
	\begin{itemize}
		#1
	\end{itemize}
}

\newcommand{\slide}[2]{
	\begin{frame}{#1}
		#2
	\end{frame}
}

%% Document

\title{Web Components Introduction}
\subtitle{A quick guide on how to use Web Components}
\author{\href{https://www.fihlon.ch/}{Marcus Fihlon}}
\institute[Fihlon]{\href{https://www.fihlon.ch/}{Scrum Master | Software Engineer | Lecturer | Speaker}}
\date{\today}

\begin{document}

\maketitle
\newlength\someheight

\slide{You don't need to take pictures of the slides!}{
	\myfig{papal_election.jpg}{1}
	\tiny{Michael Sohn / AP}
}

\slide{About Me}{
	\begin{columns}
    	\begin{column}{5cm}
		\slideItems{
			\item Scrum Master
			\item Software Engineer
			\item Lecturer
			\item Speaker
		}
    	\end{column}
	    \begin{column}{5cm}
        	\myfig{McPringle}{0.8}
    	\end{column}
	\end{columns}
	\href{https://www.fihlon.ch}{www.fihlon.ch} |
	\href{https://github.com/McPringle}{github.com/McPringle} |
	\href{http://hackergarten.net}{hackergarten.net}
}

\slide{Agenda}{
	\setcounter{tocdepth}{1}
	\tableofcontents
}

\section{Intro}

\slide{Intro}{
	``Web Components are a set of standards currently being produced by Google engineers as a W3C specification that allow for the creation of reusable widgets or components in web documents and web applications. The intention behind them is to bring component-based software engineering to the World Wide Web. The components model allows for encapsulation and interoperability of individual HTML elements.'' \\
	\hfill\tiny{\href{https://en.wikipedia.org/wiki/Web_Components}{Wikipedia}}
}

\slide{Intro}{
	\slideItems{
		\item New W3C Standard
		\item Allows reuse of components
		\item The standard is divided into four specifications:
			\slideItems{
				\item Templates
				\item Shadow DOM
				\item Custom Elements
				\item Imports
			}
		\item A Web Component uses well-known technologies:
			\slideItems{
				\item HTML
				\item CSS
				\item JavaScript
			}
		\item No need of a framework or library
			\slideItems{
				\item Except an optional polyfill to support older browsers
			}
	}
}

\section{Specifications}

\subsection{Templates}

\slide{Templates}{
	\slideItems{
		\item Defines HTML parts to be reused any number of times
		\item Define reusable parts directly inside of HTML documents
		\item Is defined by the new \code{<template>} tag
		\item Can be added to the DOM using JavaScript
		\item Unlimited number of templates possible
	}
	\listing{template.html}
}

\subsection{Shadow DOM}

\slide{Shadow DOM}{
	\slideItems{
		\item Create an independent sub-DOM
		\item Not accessable from ``outside'' of the sub-DOM
		\item Avoids DOM collisions between components
		\item No side-effects of CSS or JavaScript between components
		\item Can be added to the DOM using JavaScript
		\item Unlimited number of Shadow DOMs possible
	}
	\myfig{shadow_dom.png}{.6}
}

\subsection{Custom Elements}

\slide{Custom Elements}{
	\slideItems{
		\item Connect template and shadow DOM
		\item Define reusable components
		\item Create own tags to produce readable HTML \\ \ding{213} own tags need to include a hyphen
		\item Apply styles inside of the custom element
		\item Use JavaScript for interaction
		\item Throws lifecycle events: \\ \ding{213} created, ready, attached, detached, attributeChanged
	}
	\listing{custom_element.html}
	\myfig{custom_element.png}{.5}
}

\subsection{Imports}

\slide{Imports}{
	\slideItems{
		\item Outsourcing of HTML parts
		\item Create own HTML files for components (higher reusability)
		\item Add components to HTML documents using imports
	}
	\listing{import.html}
	\myfig{import.png}{.5}
}

\section{Goodies}

\slide{CSS Variables}{
	\listing{css-variables.css}
}

\slide{CSS Mixins}{
	\listing{css-mixins.css}
}

\section{Status}

\slide{Status}{
	\begin{center}
		\begin{tabular}{ l || c | c | c | c | c }
			& \tiny{Chrome} & \tiny{Opera} & \tiny{Firfox} & \tiny{Safari} & \tiny{IE/Edge} \\
			\hline
			\hline
			\tiny{Templates} & \textcolor{green}{\ding{52}} & \textcolor{green}{\ding{52}} & \textcolor{green}{\ding{52}} & \textcolor{green}{\ding{52}} & \textcolor{green}{\ding{52}} \\
			\tiny{Shadow DOM} & \textcolor{green}{\ding{52}} & \textcolor{green}{\ding{52}} & \textcolor{yellow}{\ding{115}} & \textcolor{yellow}{\ding{115}} & \textcolor{yellow}{\ding{115}} \\
			\tiny{Custom Elements} & \textcolor{green}{\ding{52}} & \textcolor{green}{\ding{52}} & \textcolor{yellow}{\ding{115}} & \textcolor{yellow}{\ding{115}} & \textcolor{yellow}{\ding{115}} \\
			\tiny{Imports} & \textcolor{green}{\ding{52}} & \textcolor{green}{\ding{52}} & \textcolor{yellow}{\ding{115}} & \textcolor{red}{\ding{56}} & \textcolor{red}{\ding{56}} \\
			\hline
			\hline
		\end{tabular}
	\end{center}
	
	Polyfills
	\listing{polyfills.sh}
	
	Libraries
	\slideItems{
		\item \href{https://www.polymer-project.org/}{Polymer}
		\item \href{https://x-tag.github.io/}{X-Tag}
		\item \href{https://bosonic.github.io/}{Bosonic}
	}
}

\section{Live Coding}

\slide{Screenshot of Demo Application}{
	\myfig{demo_app.png}{1}
}

\slide{Components of Demo Application}{
	\myfig{demo_components.png}{1}
}

\slide{Component Structure of Demo Application}{
	\myfig{demo_structure.png}{0.9}
}

\section{Wrap-up}

\slide{Conclusion}{
	Web Components…
	\slideItems{
		\item are \bb{declarative} and \bb{reuseable}
		\item are \bb{combinable} and \bb{extensible}
		\item are \bb{interoperational} -- DOM = common demoninator
		\item allow \bb{encapsulation} -- scoping
		\item increase \bb{productivity} and \bb{accessibility}
		\item are \bb{standard}
		\item support \bb{thinking in components}
	}
}

\slide{The End}{
	\begin{center}
		\begin{huge}\bb{Thank You! Questions?}\end{huge}
		
		\qrcode[hyperlink,height=5cm]{http://bit.ly/html-wc}
		
		\url{http://bit.ly/html-wc}
	\end{center}
}

\end{document}
