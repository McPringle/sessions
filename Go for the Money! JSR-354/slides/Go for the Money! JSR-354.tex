% !TEX TS-program = xelatex
% !TEX encoding = UTF-8 Unicode
% !BIB TS-program = biber

\documentclass{beamer}
\usepackage[cm-default,no-math]{fontspec}
\usepackage{xunicode}
\usepackage{xltxtra}
\usepackage[utf8x]{inputenc}
\usepackage{listings}
\usepackage{hyperref} 
\usepackage{adjustbox}
\usepackage{tcolorbox}
\usepackage{pdfpages}
\usepackage{multicol}
\usepackage{cancel}
\usepackage[autostyle=true,german=quotes]{csquotes}
\usepackage{pifont}
\usepackage{color}
\usepackage{qrcode}

\usetheme{m} % Use metropolis theme


\definecolor{pblue}{rgb}{0.13,0.13,1}
\definecolor{pgreen}{rgb}{0,0.5,0}
\definecolor{pred}{rgb}{0.9,0,0}
\definecolor{pgrey}{rgb}{0.46,0.45,0.48}

\usepackage{listings}
\lstset{language=Java,
  frame=single,
  showspaces=false,
  showtabs=false,
  breaklines=true,
  numbers=left,
  showstringspaces=false,
  breakatwhitespace=true,
  commentstyle=\color{pgreen},
  keywordstyle=\color{pblue},
  tabsize=4,
  xleftmargin=8pt,
  stringstyle=\color{pred},
  basicstyle=\footnotesize\ttfamily,
  moredelim=[il][\textcolor{pgrey}]{$$},
  moredelim=[is][\textcolor{pgrey}]{\%\%}{\%\%}
}

\setbeamertemplate{itemize items}[square]
\setbeamercovered{transparent}

%% Commands

\newcommand{\code}[1]{\texttt{#1}}

\newcommand{\listing}[1]{
	\begin{itemize}
		\item[]\lstinputlisting[]{listings/#1}
	\end{itemize}
}

\graphicspath{ {./images/} }
\newcommand{\myfig}[2]{
	\begin{minipage}[c]{\textwidth}
		\begin{center}
			\includegraphics[keepaspectratio,width=#2\textwidth]{#1}
		\end{center}
		\vspace{3mm}
	\end{minipage}
}

\newcommand{\bb}[1]{\textbf{#1}}

\newcommand{\slideItems}[1]{
	\begin{itemize}
		#1
	\end{itemize}
}

\newcommand{\slide}[2]{
	\begin{frame}{#1}
		#2
	\end{frame}
}

%% Document

\title{Go for the Money! JSR-354}
\subtitle{A quick introduction to the Java Money and Currency API}
\author{\href{https://www.fihlon.ch/}{Marcus Fihlon}}
\institute[Fihlon]{\href{https://www.fihlon.ch/}{Scrum Master | Software Engineer | Lecturer | Speaker}}
\date{\today}

\begin{document}

\maketitle
\newlength\someheight


\slide{Notes and Photos}{
	\begin{center}
		You can take notes and photos if you want. \\
		Or you can focus on the presentation and live coding.
		
		All my slides and source files are available online. \\
		A link and a QR code are on the last slide.
	\end{center}
}

\slide{About Me}{
	\begin{columns}
    	\begin{column}{5cm}
			\begin{tiny}
				\slideItems{
					\setlength{\itemsep}{12pt}
					\item
						\begin{normalsize}\bb{Scrum Master}\end{normalsize} \\
						\href{https://www.css.ch/}{CSS Insurance}
					\item
						\begin{normalsize}\bb{Software Engineer}\end{normalsize} \\
						\href{https://www.css.ch/}{CSS Insurance} /
						\href{https://github.com/McPringle}{Open Source Software}
					\item
						\begin{normalsize}\bb{Lecturer}\end{normalsize} \\
						\href{http://www.teko.ch/}{TEKO Swiss Technical College}
					\item
						\begin{normalsize}\bb{Speaker}\end{normalsize} \\
						\href{https://www.fihlon.ch/}{Conferences / User Groups / Meetups}
				}
			\end{tiny}
    	\end{column}
	    \begin{column}{5cm}
        	\myfig{McPringle}{0.8}
    	\end{column}
	\end{columns}
	\href{https://www.fihlon.ch}{www.fihlon.ch} |
	\href{https://github.com/McPringle}{github.com/McPringle} |
	\href{http://hackergarten.net}{hackergarten.net}
}

\slide{Agenda}{
	\setcounter{tocdepth}{1}
	\tableofcontents
}

\section{Intro}

\section{Currency and Monetary Amounts}

\section{Formatting and Parsing}

\section{Exchange rates and Conversion}

\section{Extensions}

\section{Live Coding}

\section{Wrap-up}

\slide{The End}{
	\begin{center}
		\begin{huge}\bb{Thank You! Questions?}\end{huge}
		
		\qrcode[hyperlink,height=5cm]{http://bit.ly/…}
		
		\url{http://bit.ly/…}
	\end{center}
}

\end{document}
