% !TEX TS-program = xelatex
% !TEX encoding = UTF-8 Unicode
% !BIB TS-program = biber

\documentclass{beamer}
\usepackage[cm-default,no-math]{fontspec}
\usepackage{xunicode}
\usepackage{xltxtra}
\usepackage[utf8x]{inputenc}
\usepackage{listings}
\usepackage{hyperref} 
\usepackage{adjustbox}
\usepackage{tcolorbox}
\usepackage{pdfpages}
\usepackage{multicol}
\usepackage{cancel}
\usepackage[autostyle=true,german=quotes]{csquotes}
\usepackage{pifont}
\usepackage{color}
\usepackage{qrcode}

\usetheme{m} % Use metropolis theme


\definecolor{pblue}{rgb}{0.13,0.13,1}
\definecolor{pgreen}{rgb}{0,0.5,0}
\definecolor{pred}{rgb}{0.9,0,0}
\definecolor{pgrey}{rgb}{0.46,0.45,0.48}

\usepackage{listings}
\lstset{language=Java,
  frame=single,
  showspaces=false,
  showtabs=false,
  breaklines=true,
  numbers=left,
  showstringspaces=false,
  breakatwhitespace=true,
  commentstyle=\color{pgreen},
  keywordstyle=\color{pblue},
  tabsize=4,
  xleftmargin=8pt,
  stringstyle=\color{pred},
  basicstyle=\footnotesize\ttfamily,
  moredelim=[il][\textcolor{pgrey}]{$$},
  moredelim=[is][\textcolor{pgrey}]{\%\%}{\%\%}
}

\setbeamertemplate{itemize items}[square]
\setbeamercovered{transparent}

%% Commands

\newcommand{\code}[1]{\texttt{#1}}

\newcommand{\listing}[1]{
	\begin{itemize}
		\item[]\lstinputlisting[]{listings/#1}
	\end{itemize}
}

\graphicspath{ {./images/} }
\newcommand{\myfig}[2]{
	\begin{minipage}[c]{\textwidth}
		\begin{center}
			\includegraphics[keepaspectratio,width=#2\textwidth]{#1}
		\end{center}
		\vspace{3mm}
	\end{minipage}
}

\newcommand{\bb}[1]{\textbf{#1}}

\newcommand{\slideItems}[1]{
	\begin{itemize}
		#1
	\end{itemize}
}

\newcommand{\slide}[2]{
	\begin{frame}{#1}
		#2
	\end{frame}
}

%% Document

\title{Go for the Money! JSR-354}
\subtitle{A quick introduction to the Java Money and Currency API}
\author{\href{https://www.fihlon.ch/}{Marcus Fihlon}}
\institute[Fihlon]{\href{https://www.fihlon.ch/}{Scrum Master | Software Engineer | Lecturer | Speaker}}
\date{\today}

\begin{document}

\maketitle
\newlength\someheight

\slide{Disclaimer}{
	\begin{small}
		The following presentation has been approved for open audiences only. Hypersensitivity to occasional profanity requires covering ears.
		
		All logos, photos etc. used in this presentation are the property of their respective copyright owners and are used here for educational purposes only. Any and all marks used throughout this presentation are trademarks of their respective owners.
		
		The presenter is not acting on behalf of CSS Insurance, neither as an official agent nor representative. The views expressed are those solely of the presenter.
		
		Marcus Fihlon disclaims all responsibility for any loss or damage which any person may suffer from reliance on this information or any opinion, conclusion or recommendation in this presentation whether the loss or damage is caused by any fault or negligence on the part of presenter or otherwise.
	\end{small}
}

\slide{Notes and Photos}{
	\begin{center}
		You can take notes and photos if you want. \\
		Or you can focus on the presentation and live coding.
		
		All my slides and source files are available online. \\
		A link and a QR code are on the last slide.
	\end{center}
}

\slide{About Me}{
	\begin{columns}
    	\begin{column}{5cm}
			\begin{tiny}
				\slideItems{
					\setlength{\itemsep}{12pt}
					\item
						\begin{normalsize}\bb{Scrum Master}\end{normalsize} \\
						\href{https://www.css.ch/}{CSS Insurance}
					\item
						\begin{normalsize}\bb{Software Engineer}\end{normalsize} \\
						\href{https://www.css.ch/}{CSS Insurance} /
						\href{https://github.com/McPringle}{Open Source Software}
					\item
						\begin{normalsize}\bb{Lecturer}\end{normalsize} \\
						\href{http://www.teko.ch/}{TEKO Swiss Technical College}
					\item
						\begin{normalsize}\bb{Speaker}\end{normalsize} \\
						\href{https://www.fihlon.ch/}{Conferences / User Groups / Meetups}
				}
			\end{tiny}
    	\end{column}
	    \begin{column}{5cm}
        	\myfig{McPringle}{0.8}
    	\end{column}
	\end{columns}
	\href{https://www.fihlon.ch}{www.fihlon.ch} |
	\href{https://github.com/McPringle}{github.com/McPringle} |
	\href{http://hackergarten.net}{hackergarten.net}
}

\slide{Agenda}{
	\setcounter{tocdepth}{1}
	\tableofcontents
}

\section{Intro}

\slide{Money}{
	\begin{quote}
		A large proportion of the computers in this world manipulate money, so it’s always puzzled me that money isn’t actually a first class data type in any mainstream programming language. The lack of a type causes problems, the most obvious surrounding currencies…
	\end{quote}
	\tiny{\href{http://martinfowler.com/eaaCatalog/money.html}{Martin Fowler}}
}

\slide{Before}{
	\slideItems{
		\item \code{float}, \code{double} (since Java 1)
		\item \code{java.math.BigDecimal} (since Java 1.1)
		\item \code{java.text.DecimalFormat} (since Java 1.1)
		\item \code{java.util.Currency} (since Java 1.4)
	}
}

\slide{Motivation}{
	\slideItems{
		\item Monetary values are a key feature for many applications
		\item \code{java.util.Currency} is a structure for ISO-4217 only
		\item No standard value type to represent a monetary amount
		\item No support for currency arithmetic or conversion
		\item No support for historic or virtual currencies
		\item \code{java.text.DecimalFormat} lacks flexibility
	}
}

\slide{Requirements}{
	\slideItems{
		\item Easy \bb{addition} and \bb{modification} of currencies
		\item Currencies need a \bb{context} and should be \bb{client-aware}
		\item Standard API for \bb{money amounts}, \bb{rounding}, \bb{conversions}
		\item Easy, flexible, complex, individual \bb{formatting} and \bb{parsing}
		\item Clearly defined \bb{extension points}
		\item Follow the \bb{design principles} of the Java platform
		\item Compatibility with \bb{Standard Edition} and \bb{Micro Edition}
		\item No external \bb{dependencies}
		\item \bb{Interoperability} with existing artifacts
		\item Support \bb{functional} programming style
	}
}

\slide{Four years of hard work}{
	\slideItems{
		\item JSR-354 started early 2012
		\item JSR-354 early draft review in 2013
		\item Reference implementation startet late 2013
		\item Final release of JSR-354 early 2015
		\item Final release of reference implementation early 2015
	}
}

\section{API}

\slide{Currencies}{
	\slideItems{
		\item \code{Monetary}
			\slideItems{
				\item Get a currency by code or locale
				\item Additional API for complex queries
				\item Supports SPI to enhance functionality
			}
		\item \code{CurrencyUnit}
			\slideItems{
				\item Currency code (string and number)
				\item Additional context information (e.g. type, capabilities)
			}
		\item \code{CurrencyQueryBuilder}
			\slideItems{
				\item Builder for complex queries for accesing currency units
			}
	}
}

\slide{Monetary Amounts}{
	\slideItems{
		\item \code{Monetary}
			\slideItems{
				\item Get a monetary amount by currency and value
				\item Optionally specify an explicit factory
				\item Or query for a suitable factory
				\item Supports SPI to enhance functionality
			}
		\item \code{MonetaryAmount}
			\slideItems{
				\item Numeric value and currency
				\item Arithmetic operations to do calculations
				\item Multiple implementations
				\item Interoperability rules
				\item Additional context information (e.g. capabilities)
			}
		\item \code{MonetaryAmountFactoryQueryBuilder}
			\slideItems{
				\item Builder for complex queries for accesing monetary amout factories
			}
	}
}

\slide{Rounding}{
	\slideItems{
		\item \code{Monetary}
			\slideItems{
				\item Get a rounding operator
				\item Optionally specify a locale
				\item Or query for a suitable rounding operator
				\item Supports SPI to enhance functionality
			}
		\item \code{MonetaryRounding}
			\slideItems{
				\item Extends \code{MonetaryOperator}
				\item Multiple implementations
				\item Additional context information
			}
		\item \code{RoundingQueryBuilder}
			\slideItems{
				\item Builder for complex queries for accesing rounding operators
			}
	}
}

\slide{Conversion with currency conversion}{
	\slideItems{
		\item \code{MonetaryConversions}
			\slideItems{
				\item Get a currency conversion by currency code or unit
				\item Or query for a suitable currency conversion
				\item Supports SPI to enhance functionality
			}
		\item \code{CurrencyConversion}
			\slideItems{
				\item Multiple implementations
				\item Source and target currency
				\item Conversion factor
				\item Additional context information
				\item Unidirectional
			}
		\item \code{ConversionQueryBuilder}
			\slideItems{
				\item Builder for complex queries for accesing currency conversions
			}
	}
}

\slide{Conversion with exchange rate provider}{
	\slideItems{
		\item \code{MonetaryConversions}
			\slideItems{
				\item Get an exchange rate provider
				\item Or query for a suitable currency conversion
				\item Supports SPI to enhance functionality
			}
		\item \code{ExchangeRateProvider} $\rightarrow$ \code{ExchangeRate}
			\slideItems{
				\item Multiple implementations
				\item Source and target currency
				\item Conversion factor
				\item Additional context information
				\item Unidirectional
			}
		\item \code{ConversionQueryBuilder}
			\slideItems{
				\item Builder for complex queries for accesing exchange rate providers
			}
	}
}

\slide{Formatting and parsing}{
	\slideItems{
		\item \code{MonetaryFormats}
			\slideItems{
				\item Get a monetary amount format
				\item Optionally specify a locale
				\item Or query for a suitable monetary amount format
				\item Supports SPI to enhance functionality
			}
		\item \code{MonetaryAmountFormat}
			\slideItems{
				\item Multiple implementations
				\item Additional context information
				\item Format monetary amounts
				\item Parse monetary amounts
			}
		\item \code{AmountFormatQueryBuilder}
			\slideItems{
				\item Builder for complex queries for accesing monetary amount formats
			}
	}
}

\section{Code}

\slide{}{
	\begin{center}
		\huge{Code}
	
		\large{Demo Application Walkthrough}
	\end{center}
}

\section{Wrap-up}

\slide{Validation}{
	\href{https://github.com/zalando/money-validation}{Money Validation by Zalando}
	\slideItems{
		\item Validate monetary amounts
		\item Uses existing, standardized constraints
		\item Offers additional, more expressive custom constraints
		\item Can be use with any \href{http://beanvalidation.org/}{Bean Validation} implementation
	}
	\listing{validation.java}
}

\slide{Compatibility}{
	\slideItems{
		\item Supports
			\slideItems{
				\item Java Micro Edition
				\item Java Standard Edition
				\item Java Enterprise Edition
			}
		\item Compatible with Java 8+
		\item Backport available for Java 7
	}
}

\slide{Maven}{
	\code{pom.xml}
	\listing{pom.xml}
}

\slide{Gradle}{
	\code{build.gradle}
	\listing{build.gradle}
}

\slide{Links}{
	\slideItems{
		\item Java Money Umbrella Site \\ \url{http://javamoney.org/}
		\item JSR-354 Specification \\ \url{http://javamoney.github.io/api.html}
		\item JSR-354 Reference Implementation \\ \url{http://javamoney.github.io/ri.html}
		\item JSR-354 Technical Compatibility Kit \\ \url{http://javamoney.github.io/tck.html}
		\item Java Money Financial Library \\ \url{http://javamoney.github.io/lib.html}
		\item Money Validation \\ \url{https://github.com/zalando/money-validation}
	}
}

\slide{The End}{
	\begin{center}
		\begin{huge}\bb{Thank You! Questions?}\end{huge}
		
		\qrcode[hyperlink,height=5cm]{http://bit.ly/jsr-354}
		
		\url{http://bit.ly/jsr-354}
	\end{center}
}

\end{document}
